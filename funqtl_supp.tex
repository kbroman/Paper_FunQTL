\documentclass[letterpaper,twoside]{article}

\usepackage{caption}

\usepackage[authoryear]{natbib}
\usepackage{times}
\usepackage{amsmath,amsfonts}
\usepackage{graphicx}
\usepackage{indentfirst}
\usepackage{color}
\usepackage{url}
\usepackage[paper=letterpaper]{geometry}
\usepackage{fontspec}
\setmainfont[Mapping=text-tex]{Calibri}

% revise margins
\setlength{\headheight}{0.25in}
\setlength{\headsep}{-0.25in}
\setlength{\topmargin}{0.0in}
\setlength{\textheight}{9in}
\setlength{\footskip}{0.5in}
\setlength{\oddsidemargin}{-0.25in}
\setlength{\evensidemargin}{-0.25in}
\setlength{\textwidth}{7.0in}
\setlength{\rightskip}{0pt plus 1fil} % makes ragged right

% headings
\usepackage{fancyhdr}
\pagestyle{fancy}
\fancyfoot{}
\fancyfoot[C]{I.-Y.\ Kwak \emph{et al.}}
\fancyfoot[RO, LE]{\textbf{\thepage\ SI}}
\fancyhead{}
\renewcommand{\headrulewidth}{0pt}
\renewcommand{\footrulewidth}{0pt}

\captionsetup[figure]{labelsep=quad}
\captionsetup[table]{labelsep=quad}

% make figures S1, S2, ...
\renewcommand{\thefigure}{\textbf{S\arabic{figure}}}
\renewcommand{\figurename}{\textbf{Figure}}

% make tables S1, S2, ...
\renewcommand{\thetable}{\textbf{S\arabic{table}}}
\renewcommand{\tablename}{\textbf{Table}}

\begin{document}

\vspace*{8mm}
\begin{center}

\textbf{\Large A simple regression-based method to map \\[6pt]
  quantitative trait loci underlying function-valued
  phenotypes}

\bigskip \bigskip \bigskip \bigskip

\textbf{\Large SUPPLEMENT}

\bigskip \bigskip
\bigskip \bigskip


{\large Il-Youp Kwak$^*$, Candace R. Moore$^\dagger$, Edgar
  P. Spalding$^\dagger$, Karl W. Broman$^{\ddagger}$}

\bigskip \bigskip

Departments of $^*$Statistics, $^\dagger$Botany, and $^\ddagger$Biostatistics and Medical
Informatics, \\
University of Wisconsin--Madison, Madison, Wisconsin 53706
\end{center}




\clearpage

\begin{figure}[p]
\centerline{\includegraphics{Figs/figS1.eps}}

\caption{Genetic map of typed genetic markers (A)
     and function-valued phenotypes for five randomly selected
     Arabidopsis RIL (B), for data from Edgar Spalding and colleagues.}
\end{figure}

\clearpage

\begin{figure}[p]
\centerline{\includegraphics{Figs/figS2.eps}}
\vspace{1cm}

\caption{Average (A) and standard deviation (B) of the root tip angle phenotype at
  each individual time point, and the correlations between time points (C).}
\end{figure}

\clearpage

\begin{figure}[!ht]
\begin{center}
\includegraphics{Figs/figS3.eps}
\vspace{1cm}
 \caption{Growth curves for the three QTL genotypes in the single-QTL
   simulation study.}
\end{center}
\end{figure}

\clearpage

\begin{figure}[!ht]
\begin{center}
\includegraphics{Figs/figS4.eps}
\vspace{1cm}
 \caption{The heritability for each time point in the single-QTL
   simulation study, for the three assumed variance structures and the
   chosen values for the $c$ parameter.}
\end{center}
\end{figure}

\newpage

\begin{figure}[!ht]
\begin{center}
\includegraphics{Figs/figS5.eps}
\vspace{1cm}
 \caption{Root Mean Square Error (RMSE) of the estimated
   QTL position as a function of the percent variance explained by a
   single QTL.  The first column is for $n=100$, the second column is for $n=200$ and the third column is
   for $n=400$. The three rows correspond to the covariance structure
   (autocorrelated, equicorrelated, and unstructured).  In each panel,
   SLOD is in red, MLOD is in blue, EE(Wald) is in brown, EE(Residual)
   is in green, and parametric is in black.}
\end{center}
\end{figure}

\newpage

\begin{figure}[!ht]
\begin{center}
\includegraphics{Figs/figS6.eps}
\vspace{1cm}
 \caption{Root Mean Square Error (RMSE) of the estimated QTL position
   as a function of the percent variance explained by a single QTL,
   with additional noise added to the phenotypes.  The first column
   has no additional noise; the second and third columns have
   independent normally distributed noise added at each time point,
   with standard deviation 1 and 2, respectively. The three rows
   correspond to the covariance structure (autocorrelated,
   equicorrelated, and unstructured). In each panel, SLOD is in red,
   MLOD is in blue, EE(Wald) is in brown, EE(Residual) is in green,
   and parametric is in black. The percent variance explained by the
   QTL on the x-axis refers, in each case, to the variance explained
   in the case of no added noise.}
\end{center}
\end{figure}

\clearpage

\begin{figure}[!ht]
\begin{center}
\includegraphics{Figs/figS7.eps}
\vspace{1cm}
 \caption{The underlying true baseline function (A) and
   QTL effect curves (B, C and D) for the multiple-QTL simulations.}
\end{center}
\end{figure}




\clearpage

\begin{table}[p]

\centering
  \caption{5\% significance thresholds for the data from Moore \emph{et
    al.} 2013, \\ based on a permutation test with 1000 replicates.}

\bigskip

    \begin{tabular}{l@{\hspace{6mm}}cr@{.}lc}
      \hline
      \textbf{Method} & \multicolumn{4}{c}{\textbf{Threshold}} \\
      \hline
      SLOD         &\hspace{1mm} & 1&85   &\hspace{1mm} \\
      MLOD         && 3&32   &\\
      EE(Wald)     && 5&72   &\\
      EE(Residual) && 0&0559 &\\
      \hline
    \end{tabular}
\end{table}


\clearpage


\begin{table}[p]
\centering
  \caption{The unstructured covariance matrix used in the single-QTL simulations.}
\begin{equation*}
  \Sigma = \begin{pmatrix}
0.72 & 0.39 & 0.45 & 0.48 & 0.50 & 0.53 & 0.60 & 0.64 & 0.68 & 0.68 \\
0.39 & 1.06 & 1.61 & 1.60 & 1.50 & 1.48 & 1.55 & 1.47 & 1.35 & 1.29 \\
0.45 & 1.61 & 3.29 & 3.29 & 3.17 & 3.09 & 3.19 & 3.04 & 2.78 & 2.53 \\
0.48 & 1.60 & 3.29 & 3.98 & 4.07 & 4.01 & 4.17 & 4.18 & 4.00 & 3.69 \\
0.50 & 1.50 & 3.17 & 4.07 & 4.70 & 4.68 & 4.66 & 4.78 & 4.70 & 4.36 \\
0.53 & 1.48 & 3.09 & 4.07 & 4.68 & 5.56 & 6.23 & 6.87 & 7.11 & 6.92 \\
0.60 & 1.55 & 3.19 & 4.17 & 4.66 & 6.23 & 8.59 & 10.16 & 10.80 & 10.70 \\
0.64 & 1.47 & 3.04 & 4.18 & 4.78 & 6.87 & 10.16 & 12.74 & 13.80 & 13.80 \\
0.68 & 1.35 & 2.78 & 4.00 & 4.70 & 7.11 & 10.80 & 13.80 & 15.33 & 15.35 \\
0.68 & 1.29 & 2.53 & 3.69 & 4.36 & 6.92 & 10.70 & 13.80 & 15.35 & 15.77 \\
  \end{pmatrix}
\end{equation*}
\end{table}

\end{document}
